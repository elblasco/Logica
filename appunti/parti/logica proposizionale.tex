\documentclass[../main.tex]{subfiles}
\usepackage[margin=1.5cm]{geometry}
\usepackage{hyperref}
\usepackage{qtree}
\usepackage{amsmath}
\usepackage{makecell}
\usepackage{listings}

\newcommand{\spazio}{\vspace{1em} \newline}

\begin{document}
    \chapter{La logica}
    \section{Intuizioni iniziali}
    Le caratteristiche della logica proposizionale sono:
    \begin{itemize}
        \item Un linguaggio proposizionale contiene solo proposizioni primitive.
        \item Le formule sono interpretate attraverso un dominio di giudizi.
        \item Le formule complesse sono formate usando un numero arbitrario di connettivi proposizionali.
        \item I connettivi proposizionali possono essere.
        \begin{itemize}
            \item $\lnot$ letto come "not" per la negazione.
            \item $\land$ letto come "and" per le congiunzioni.
            \item $\lor$ letto come "or" per le disgiunzioni.
            \item $\Longrightarrow$ letto come "implies" per le implicazioni.
            \item $\Longleftrightarrow$ letto come "if and only if" per le equivalenze.
            \item $\uparrow$ letto come "nand" per le congiunzioni negate.
            \item $\downarrow$ letto come "nor" per le disngiunzioni negate.
        \end{itemize} 
    \end{itemize}
    La logica proposizionale è utile per problemi che possono essere formalizzati per essere indipendenti da strutture interne.\\
    Sulla logica proposizionale possiamo fare le seguenti osservazioni:
    \begin{itemize}
        \item Una proposizione è una frase che descrive il mondo.
        \item Una proposizione può essere o vera o falsa.
        \item "not P" è vera se P è falsa e viceversa.
        \item "P and Q" è vera se e solo se P e Q sono entrambe vere.
        \item "P or Q" per essere vera è sufficiente che solo una delle due sia vera.
        \item "P implies Q" indica che Q è vera quando lo è P, ma non dice nulla su quando P è falsa.
        \item "P if and only if Q" indica che P e Q devo essere vere/false allo stesso momento.
        \item "P xor Q" è vera solo se una delle due è vera.
    \end{itemize}

    \section{Sintassi}
    \subsection{Alfabeto proposizionale}
    L'alfabeto è composto da:
    \begin{itemize}
        \item Simboli logici: $\lnot, \land, \lor,\supset , \equiv$.
        \item Simboli non logici: costanti proposizionali e insiemi \textbf{PROP} che contengono simboli P chiamati variabili proposizionali che possono contenere una costante proposizionale come un valore.
        \item Simboli separatori: "(" e ")".
    \end{itemize}

    \subsection{Formule ben formate(wff)}
    Una formula wff viene definita come segue:
    \begin{itemize}
        \item Ogni P$\in$\textbf{PROP} è una formula atomica.
        \item Ogni formula atomica è wff.
        \item Se A e B sono formule, allora $\lnot$A, A$\land$B, A$\lor$B, A$\supset$B e A$\equiv$B sono formule.
    \end{itemize}
    Per leggere una formula è importante sapere che i vari operatori hanno delle diverse priorità, come in matematica le parentesi fungono da modificatore delle priorità.
    \begin{center}
        \begin{tabular}{c|c}
            \textbf{Simbolo} & \textbf{Priorità}\\
            \hline
            $\lnot$ & 1\\
            $\land$ & 2\\
            $\lor$ & 3\\
            $\supset$ & 4\\
            $\equiv$ & 5\\
        \end{tabular}
    \end{center}

    \subsection{Sottoformule}
    Una sottoformula di una formula, rappresentata come un albero, indica l'insieme di tutti i suoi sottoalberi.\\
    Viene formalmente definita come:
    \begin{itemize}
        \item A è una sottoformula di se stessa.
        \item A e B sono sottoformule di A$\land$B, A$\lor$B, A$\supset$B e A$\equiv$B.
        \item A è una sottoformula di $\lnot$A.
        \item Se A è una sottoformula di B e B è una sottoformula di C, allora A è una sottoformula di C.
        \item A è una sottoformula propria di B se A è sottoformula di B e A$\neq$B.
    \end{itemize}

    \subsubsection{Esempi}
    \begin{enumerate}
        \item Se piove mentre splende il sole apparirà l'arcobaleno.\\
            p=piove; q=splende il sole; r=arcobaleno;\\
            (p$\land$q)$\supset$r
        \item Claudio viene se Elsa viene.\\
            p=Claudio viene; q=Elsa viene;\\
            p$\supset$q
        \item Claudio viene se Elsa viene e viceversa.\\
            p=Claudio viene; q=Elsa viene;\\
            p$\equiv$q
        \item Giacomo viene se e solo se Pietro resta a casa.\\
            p=Giacomo viene; q=Pietro sta a casa;\\
            p$\equiv$q
        \item Noi andiamo se non piove.\\
            p=Noi andiamo; q=piove;\\
            p$\equiv \lnot$q
        \item Claudio ed Elsa sono o fratello e sorella o nipoti.\\
            p=Claudio ed Elsa sono fratello e sorella; q= Claudio ed Elsa sono nipoti;\\
            p$\lor$q
        \item Se perdo se non posso fare una mossa, allora ho perso.\\
            p=ho perso; q=non posso fare una mossa;\\
            (q$\supset$p)$\supset$p
    \end{enumerate}

    \section{Funzione di interpretazione}
    Il dominio di interpretazione della logica proposizionale è D=\{True, False\}.\\
    L'interpretazione proposizionale è una funzione:
    \begin{center}
        I: \textbf{PROP} $\to$ D
    \end{center}
    Se $|$\textbf{PROP}$|$ è la cardinalità di \textbf{PROP} allora esistono $2^{|\textbf{PROP}|}$ interpretazioni differenti corrispondenti a tutti i sottoinsiemi di \textbf{PROP}.

    \section{Implicazione}
    Diciamo che una funzione di interpretazione implica una formula A se:
    \begin{itemize}
        \item I$\models$A, se I(A)=True con A$\in$\textbf{PROP}.
        \item I$\models \lnot$A, se non è vero che I$\models$A.
        \item I$\models$A$\land$B se I$\models$A e I$\models$B.
        \item I$\models$A$\lor$B se I$\models$A o I$\models$B.
        \item I$\models$A$\supset$B se I$\models$A allora I$\models$B.
        \item I$\models$A$\equiv$B se I$\models$A se e solo se I$\models$B.
    \end{itemize}
    \begin{minipage}{0.5\textwidth}
        \begin{tabular}{|c|c|}
            \hline
            $\lnot$True & False\\
            $\lnot$False & True\\
            \hline
            True $\land$ True & True\\
            True $\land$ False & False\\
            False $\land$ True & False\\
            False $\land$ False & False\\
            \hline
            True $\lor$ True & True\\
            True $\lor$ False & True\\
            False $\lor$ True & True\\
            False $\lor$ False & False\\
            \hline
        \end{tabular}
    \end{minipage}
    \begin{minipage}{0.5\textwidth}
        \begin{tabular}{|c|c|}
            \hline
            True $\supset$ True & True\\
            True $\supset$ False & False\\
            False $\supset$ True & True\\
            False $\supset$ False & True\\
            \hline
            True $\equiv$ True & True\\
            True $\equiv$ False & False\\
            False $\equiv$ True & False\\
            False $\equiv$ False & True\\
            \hline
        \end{tabular}
    \end{minipage}

    \section{Errori comuni}
    \begin{itemize}
        \item Noi esprimiamo le congiunzioni con molte parole oltre a "e", degli esempio posso essere "ma", "quindi ", "per tanto", \dots
        \item Alcune vole "e" non unisce due proposizioni intere ma solo due sostantivi.
        \item Alcune volte "and" unisce due aggettivi.
        \item Il modo per esprimere una disgiunzione esclusiva è (p$\lor$q)$\land \lnot$(p$\lor$q), mentre il modo per indicare che hanno valori di verità diversi è negare la loro uguaglianza $\lnot$(p$\equiv$q).
        \item \textbf{Anche se:} la frase "p anche se q" può essere tradotta in p$\land$(q$\lor \lnot$q).
    \end{itemize}

    \chapter{Calcolo}
    \section{Tabelle di verità}
    Sono il mezzo attraverso il quale generiamo tutte le possibili interpretazioni generate considerando tutte le proposioni atomiche per rispondere ai quesiti di:
    \begin{itemize}
        \item Verifica del modello.
        \item Soddisfacibilità.
        \item Validità.
        \item Insoddisfacibilità.
        \item Conseguenza logica.
        \item Equivalenza logica.
    \end{itemize}
    Per costruire una tabella di verità, con $n$ proposioni aomiche, è possibile seguire l'algoritmo:
    \begin{enumerate}
        \item Esiste una riga per ogni interpretazione, quindi ne avrò $2^n$.
        \item Le prime $n$ colonne comprendono tutte le interpretazioni, mentre l'ultima i valori di verità della formula totale.\\
            Le colonne nel mezzo contengono i valori di verità delle sottoformule della formula finale presa in considerazione.
        \item L'asseganemnto dei valori alle proposizioni atomiche inizia da sinistra verso destra, nella prima colonnna alterno T e F con periodo 1, nella seconda con periodo 2 e nella $n$ con periodo $n$.
    \end{enumerate}

    \section{Verifica del modello}
    Sia / un interpretazione applicata ad un linguaggio proposizioanel L.\\
    Possiamo verificare che una formula A$\in$L è soddisfabile da / applicando in modo ricorsivo l'algoritmo \textbf{MCHECK(I,A)}.

    \subsection{Algoritmo}
    \subsubsection{Caso base}
    A=p
    \begin{lstlisting}
        MCHECK(I$\models$p)
        if I(p)==True then
            return YES
        else
            return NO
    \end{lstlisting}

    \subsubsection{Caso ricorsivo}
    A=B$\land$C
    \begin{lstlisting}
        MCHECK(I$\models$B$\land$C)
        if MCHECK(I$\models$B) then
            return MCHECK(I$\models$C)
        else
            return NO
    \end{lstlisting}

    \noindent
    A=B$\lor$C
    \begin{lstlisting}
        MCHECK(I$\models$B$\lor$C)
        if MCHECK(I$\models$B) then
            return YES
        else
            return MCHECK(I$\models$C)
    \end{lstlisting}

    \noindent
    A=B$\supset$C
    \begin{lstlisting}
        MCHECK(I$\models$B$\supset$C)
        if MCHECK(I$\models$B) then
            return MCHECK(I$\models$C)
        else
            return YES
    \end{lstlisting}

    \noindent
    A=B$\equiv$C
    \begin{lstlisting}
        MCHECK(I$\models$B$\equiv$C)
        if MCHECK(I$\models$B) then
            return MCHECK(I$\models$C)
        else
            return $\lnot$MCHECK(I$\models$C)
    \end{lstlisting}

    \section{Soddisfacibilità}
\end{document}