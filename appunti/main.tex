\documentclass{book}
\usepackage[margin=1.5cm]{geometry}
\usepackage{hyperref}

\title{Logica 2022/2023}
\author{Blascovich Alessio\\
    \href{mailto:alessio.blascovich@studenti.unitn.it}{alessio.blascovich@studenti.unitn.it}}
\date{}

\begin{document}
    \maketitle
    \tableofcontents
    \part{Modellazione}
    \chapter{Rappresentazione}

    \section{Rappresentazione mentale}
    \begin{itemize}
        \item \textbf{Mondo:} Il mondo è ciò che assumiamo esista.
        \item \textbf{Rappresentazione mentale:} Una rappresentazione mentale è una parte del mondo che descrive il mondo stesso, quindi esiste corrispondza tra cosa esiste nel mondo e una rappresentazione mentale.\\
            La rappresentazione mentale permette di agire nel mondo e di interagire con altri umani.
        \item \textbf{Rappresentazione mentale analogica:} Una rappresentazione analogica è una semplice rappresentazione che si basa su ciò che percepiamo con i nostri sensi.
        \item \textbf{Rappresentazione linguistica mentale:} La rappresentazione che descrive il contesto di una rappresentazione analogica.\\
            Viene usata per:
            \begin{itemize}
                \item \textbf{Descrivere:} cosa è successo nella rappresentazione mentale analogica.
                \item \textbf{Comunicare:} con altri umani approposito della rappresentazione e quindi del mondo.
                \item \textbf{Imparare:} da cosa viene descritto.
                \item \textbf{Motiva:} cioè cerca di distinguere quello che non sa da quello che già conosce. 
            \end{itemize}
    \end{itemize}

    \section{Rappresentazione}
    \begin{itemize}
        \item \textbf{Rapresentazione:} Ha due proprietà principali:
            \begin{itemize}
                \item Pìù umani la percepiscono (come la rappresentazione mentale).
                \item E' una parte dello stesso mondo che descrive.
            \end{itemize}
        \item \textbf{Rappresentazione analogica:} Fa una mappatura uno a uno del mondo, considerando anche il contesto in cui ci si trova.
        \item \textbf{Linguistic representation:} Non la ho capita, ste cazzate filosofiche.
    \end{itemize}

    \chapter{Modellazione mentale}
    \section{Conoscenza semantica}
    \begin{itemize}
        \item \textbf{Teoria:} Identifica la rappresentazione linguistica prodotta da attività di modellazione.
        \item \textbf{Modello:} E' una rappresentazione analogica data da un modello.\\
            Possiamo anche dire che sia il modello inteso dalla teoria e che $T$ sia la teoria del modello $M$.
        \item \textbf{Modello mondiale:} Un modello mondiale $M_W$ è una cappia data da teoria $T$ e modello $M_T$ legati dalla formula: $M_W=\langle T, M_T \rangle$.
        \item \textbf{Denotazione, semantiche:} Diremo che la teoria $T$ denota il suo modello inteso $M$ e scriveremo $T=D(M)$..
            In modo alternativo possiamo dire che il modello $M$ è la semantica intesa da $T$ e scriveremo $M=S(T)$.
        \item \textbf{Fatto:} Un modello $M={f}$ è un insieme di fatti $f$, dove i fatti sono una rappresentazione analogica una parte della parte di mondo descritta da $M$.
        \item \textbf{Frase:} Una teoria $T={s}$ è un insieme di frasi $s$, dove una frase è una rappresentazione linguistica di un insieme di fatti $f$.
        \item \textbf{Denotazione, semantiche:} Diremo che una frase $s$ denota un fatto $f$ e scriveremo $s=D(f)$.\\
            Alternarivamente, un fatto $f$ è la semantica intesa da $s$ e scriveremo $f=S(s)$.
    \end{itemize}
\end{document}