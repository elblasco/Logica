\documentclass{book}
\usepackage[margin=1.5cm]{geometry}
\usepackage{hyperref}
\newcommand{\spazio}{\vspace{1em} \newline}

\title{Logica 2022/2023}
\author{Blascovich Alessio\\
    \href{mailto:alessio.blascovich@studenti.unitn.it}{alessio.blascovich@studenti.unitn.it}}
\date{}

\begin{document}
    \maketitle
    \tableofcontents
    \part{Modellazione}
    \chapter{Rappresentazione}

    \section{Rappresentazione mentale}
    \begin{itemize}
        \item \textbf{Mondo:} Il mondo è ciò che assumiamo esista.
        \item \textbf{Rappresentazione mentale:} Una rappresentazione mentale è una parte del mondo che descrive il mondo stesso, quindi esiste corrispondza tra cosa esiste nel mondo e una rappresentazione mentale.\\
            La rappresentazione mentale permette di agire nel mondo e di interagire con altri umani.
        \item \textbf{Rappresentazione mentale analogica:} Una rappresentazione analogica è una semplice rappresentazione che si basa su ciò che percepiamo con i nostri sensi.
        \item \textbf{Rappresentazione linguistica mentale:} La rappresentazione che descrive il contesto di una rappresentazione analogica.\\
            Viene usata per:
            \begin{itemize}
                \item \textbf{Descrivere:} cosa è successo nella rappresentazione mentale analogica.
                \item \textbf{Comunicare:} con altri umani approposito della rappresentazione e quindi del mondo.
                \item \textbf{Imparare:} da cosa viene descritto.
                \item \textbf{Motiva:} cioè cerca di distinguere quello che non sa da quello che già conosce. 
            \end{itemize}
    \end{itemize}

    \section{Rappresentazione}
    \begin{itemize}
        \item \textbf{Rapresentazione:} Ha due proprietà principali:
            \begin{itemize}
                \item Pìù umani la percepiscono (come la rappresentazione mentale).
                \item E' una parte dello stesso mondo che descrive.
            \end{itemize}
        \item \textbf{Rappresentazione analogica:} Fa una mappatura uno a uno del mondo, considerando anche il contesto in cui ci si trova.
        \item \textbf{Linguistic representation:} Non la ho capita, ste cazzate filosofiche.
    \end{itemize}

    \section{Modellazione}
    \subsection{Modellazione}
    \begin{itemize}
        \item \textbf{Modellazione:} La modellazione è l'attività che porta alla realizzazione di una rappresentazione attraverso una serie di rappresentazioni mentali intermedie.
        \item \textbf{Teoria:} Identifica la rappresentazione linguistica prodotta da attività di modellazione.
        \item \textbf{Modello:} E' una rappresentazione analogica data da un modello.\\
            Possiamo anche dire che sia il modello inteso dalla teoria e che $T$ sia la teoria del modello $M$.
        \item \textbf{Modello mondiale:} Un modello mondiale $M_W$ è una cappia data da teoria $T$ e modello $M_T$ legati dalla formula: $M_W=\langle T, M_T \rangle$.
        \end{itemize}

        \subsection{Osservazioni}
        Nella maggior parte dei modelli mondiali, usati in applicationi CS/AI, è definita solo la teoria.\\
        Il modello inteso è implicito visto che la rappresentazione mentale è simile tra le perone.\\
        Questo modello viene seguito quando il costo degli errori è accettabile.
        \spazio
        Alcune volte il modello semantico è sviluppato in seguito e non copre totalmente la semantica.
        \spazio
        Alcune volte la mancanza di un modello esplicito porta alla genesi di un dialetto.\\
        Nel caso particolare del IA questa mancanza fa compiere alla macchina azioni imprevedibili perchè impedisce alla macchina di capire quando sbaglia.

    \chapter{Modellazione mentale}
    \section{Conoscenza semantica}
    \subsection{Teoria e modelli}
    \begin{itemize}
        \item \textbf{Denotazione e semantiche:} Diremo che la teoria $T$ denota il suo modello inteso $M$ e scriveremo $T=D(M)$..
            In modo alternativo possiamo dire che il modello $M$ è la semantica intesa da $T$ e scriveremo $M=S(T)$.
    \end{itemize}

    \subsection{Frasi e fatti}
    \begin{itemize}
        \item \textbf{Fatto:} Un modello $M=\{f\}$ è un insieme di fatti $f$, dove i fatti sono una rappresentazione analogica una parte della parte di mondo descritta da $M$.
        \item \textbf{Frase:} Una teoria $T=\{s\}$ è un insieme di frasi $s$, dove una frase è una rappresentazione linguistica di un insieme di fatti $f$.
        \item \textbf{Denotazione e semantiche:} Diremo che una frase $s$ denota un fatto $f$ e scriveremo $s=D(f)$.\\
            Alternarivamente, un fatto $f$ è la semantica intesa da $s$ e scriveremo $f=S(s)$.
    \end{itemize}

    \subsection{Osservazioni}
    Assumiamo sempre che una frase $s \in T$ descriva uno o più fatti $f \in M$.\\
    La nozione della descrizione linguistica e, in particolare, quella della toeria e della frase, possiamo sempre assumere si riferisca (la nozione) a una descrizione mentale possibilmente resa oggettiva tramite un modello che la descrive.
    \spazio
    Non esistono rappresentazioni linguistiche senza referenze al mondo.
    \spazio
    Una frase $s=D(f)$ può denotare più fatti la $D$ è una relazione e non necessariamente una funzione.\\
    In questi casi diremo che $s$ è ambigua o polisemica (polisemica=esprime più significati).
    \spazio
    Un fatto $f=S(s)$ può essere la semantica di più frasi $s$, allora anche $S$ è una relazione e non per forza una funzione.\\
    Per una fatto $f$ ci sono infiti modi di denotarlo e per questo le frasi che denotano lo stesso fatto vengono dette sinonimi.
    \spazio
    Se $D$ e $S$ sono entrambe funzioni allora sono una l'inversa dell'altra.

    \subsection{Toerie e modelli}
    \begin{itemize}
        \item \textbf{Modello minore:} Siano due modelli $M=\{f\}$ e $\widehat{M}=\{f\}$ tali che $\widehat{M}\subseteq M$.\\
            Diremmo che $\widehat{M}$ è minore rispetto a $M$ e che un fatto $f$ tale che $f \in M$ e $f \notin \widehat{M}$ è detto al di fuori di $\widehat{M}$.
        \item \textbf{Teorie e modelli:} Sia $M=\{f\}$ un insieme diu fatti e $T=\{s\}$ un insieme di sequenze.\\
            Sia $M_T$ un inseieme minore di $M$, allora $T$ è una teoria del modello $M_T$ se e solo se $\forall\ s\in T$ abbiamo $s=D(f) $ per qualsiasi $f \in M_T$.\\
            Possiamo anche dire che $M_T$ è un modello di $T$. 
    \end{itemize}

    \subsection{Osservazioni}
    Esistono modelli e teorie che sono dei singoletti, ma i due eventi sono indipendenti.
    \spazio
    Modelli di fatti diversi possono rappresentare lo stesso mondo a diversi livelli di astrazione, analogamente vale anche per le teorie di frasi corrispondenti.
    \spazio
    Più il modello è astratto meno dettagli contiene rispetto a un modello meno astratto.\\
    Un modello meno astratto può comunque rappresentare una gran parte del mondo.
    \spazio
    Un modello piccolo rappresenta una piccola parte del mondo, analogamente fanno le teorie.

    \subsection{Correttezza e completezza}
    \begin{itemize}
        \item \textbf{Correttezza:} Sia $M_T \subseteq M$, allora una teoria $T$ del modello $M_T$ è corretta rispetto a $M_T$ se e solo se:\\
            $\forall\ s \in T\ \exists f \in M_T\ |\ f=S(s)$, viene detta incorretta altrimenti.
        \item \textbf{Completezza:} Sia $M_T \subseteq M$, allora una teoria $T$ di un modello $M_T$ vinene detta dette corretta rispetto a $M_T$ se e solo se $\forall\ f \in M_T\ \exists s \in T\ |\ s=D(f)$, viene detta icompleta altrimenti.
        \item \textbf{Correttezza e completezza:} Sia $M_T \subseteq M$, allora un teoria $T$ di un modello $M_T$ viene detta corretta e completa se rispetta entrambe le due condizioni.
    \end{itemize}

    \subsection{Osservazioni}
    La maggior parte delle volte una toeria risulta incompleta perchè le persone descrivono solo parte di ciò che percepiscono .
    \spazio
    La principale motivazione per l'incorreteza delle teorie è la mancanza di motivazioni.
    \spazio
    In applicazione dove il costo per errore è elevato bisogna far rispettare sia correttezza sia completezza.
    \space
    Alcune volte la completezza non è raggiungibile, perciò bisogna preferira la correttezza alla completezza.
    \space
    Solitamente il metro per preferire completezza/correttezza è dato da motivazioni pratiche come studi probabilistici.

    \section{Semantica linguistica}
    \subsection{Linguaggio e dominio}
    \begin{itemize}
        \item \textbf{Dominio:} Un dominio $D=\{M\}$ è un insieme di modelli $M$.
        \item \textbf{Linguaggio:} Un linguaggio $L=\{T\}$ è un insieme di teorie $T$.
        \item \textbf{Denotazione e semantiche:}  Diciamo che un linguaggio $L=\{T\}$ denota un dominio $D=\{M\}$ se descrive tutti i suoi modelli e scriveremo $L=Den(D)$.\\
            Possiamo anche dire che un dominio $D$ è la semantica intesa da $L$ scrivendo $D=S(L)$.
    \end{itemize}

    \subsection{Osservazioni}
    Il dominio è definito come uno spazione o qualsiasi cosa noi possiamo immaginare, cosa non possibile con i modelli in quanto devono essere una rappresentazione della realtà.
    \spazio
    Una situazione può essere modellata da diversi domini.\\
    Uno stesso dominio può essere descritto da linguaggi differenti che si concentrano su aspetti diversi.\\
    Queste due proprietà vengono dette eterogeneità semantica.
    \spazio
    Un dominio è definito come $D=\{M\}$, ma ricordando la definizione di insieme di modelli come $M=\{f\}$ con $f \in M$ e $M \in D$.\\
    Risulta che $M \subseteq D$, possiamo quindi dire che il dominio è l'insieme potenza (si, in italiano powerset è insieme potenza) dell'insieme dei fatti $\{f\}$.
    \spazio
    Un linguaggio definito $L=\{T\}$ è un insieme di modelli $T$, un linguaggio può anche essere modllato come l'insieme $L=\{s\}$ di tutte le frasi $s \in L$ appartenenti alle teorie $T \in L$.\\
    Concludiamo che $T \subseteq L$, ne risulta che un linguaggio è l'insieme potenza degll'insieme $\{s\}$.

    \section{Modellazione mentale}
    Possiamo pensare al modello mentale come formato da 4 componeneti:
    \begin{enumerate}
        \item \textbf{Linguaggio:} lo spazio di tutte le possibili teorie
        \item \textbf{Dominio:} lo spazio di tutti i possibili casi
        \item \textbf{Modello:} un insieme di fatti
        \item \textbf{Teoria:} un insieme di frasi che descrivono i fatti nel modello.
    \end{enumerate}

    \subsection{Gap semantico}
    Il mondo causa la generazione del modello mentale con i suoi 4 componenti.\\
    Il modelo mentale rappresenta sia il mondo analogico che quello linguistico, ma si discosterà sempre dal mondo per via del gap semantico.\\
    Questo gap è dato dall'imperfezione umana, dai limiti dei nostri sensi e della nostra lingua.

    \chapter{Modello mentale formale}
    \section{Linguaggio informale}
    \begin{itemize}
        \item \textbf{Termine:} è un elemento del linguaggio, il termine denota un insieme di entità del mondo.
        \item \textbf{Frase:} è un elemento del linguaggio, la frase denota un insieme di fatti.
        \item \textbf{Sintassi:} la sintassi di un linguaggio $L$, definito come $L=\{s\}$ un insieme di frasi $s$, è l'insieme di regole formali che ci permettono di definire tutte le frasi $s \in L$ partendo da primitive chiamate alfabeto.
        \item \textbf{Primita, atomica:} Un termine o frase è primitiva se appartiene all'alfabeto, si dice complesso in tutti gli altri.\\
            Una frase è atomica se è il caso base delle regole di formazione delle frasi.\\
            Frasi primitive sono atomiche ma non vale il contrario.
        \item \textbf{Sintassi 2:} la sintassi può anche essere definita come segue
            \begin{itemize}
                \item Un insieme di termini primitivi chiamati termini alfabetici.
                \item Un insieme di regola per la costruzione di termini.
                \item Un insieme di regole per la formazione termine a frase.
                \item Un insieme di frasi primitive chiamate frasi alfabetiche.
                \item Un insieme di regola per la formazione di frasi.
                \item Un insieme di regole per la formazione da frase a termine.
            \end{itemize}
    \end{itemize}

    \section{Osservazioni}
    
\end{document}