\documentclass{book}
\usepackage[margin=1.5cm]{geometry}
\usepackage{hyperref}
\usepackage{amsmath}
\usepackage{makecell}
\usepackage{listings}
\usepackage{graphicx}
\usepackage{subfiles}

\graphicspath{{img/}}

\lstset{
    basicstyle=\ttfamily,
    mathescape
}
\counterwithin{chapter}{part}

\newcommand{\spazio}{\vspace{1em} \newline}
\newcommand{\T}{\mathcal{T}}
\newcommand{\A}{\mathcal{A}}

\title{Logica 2022/2023}
\author{Blascovich Alessio\\
    \href{mailto:alessio.blascovich@studenti.unitn.it}{alessio.blascovich@studenti.unitn.it}\\
    \href{https://t.me/alessio_blascovich}{Telegram}
}
\date{}

\begin{document}
    \maketitle
    \textbf{Nota:} questo documento è stato creato per aiutare gli studenti di logica a studiare per l'esame di logica, non è un documento ufficiale dell'Università di Trento e non è stato creato da nessuno dei docenti che insegnano logica.\\
    Mi scuso per qualsiasi errore/imprecisione che potrebbe esserci, se ne trovate qualcuno, per favore contattatemi.
    \tableofcontents
    \subfile{parti/modellazione.tex}

    \subfile{parti/logica proposizionale.tex}

    \subfile{parti/logica del primo ordine.tex}

    \subfile{parti/logica descrittiva.tex}
\end{document}